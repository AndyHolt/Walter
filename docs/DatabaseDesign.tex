\documentclass[]{article}
\usepackage{lmodern}
\usepackage{amssymb,amsmath}
\usepackage{ifxetex,ifluatex}
\usepackage{fixltx2e} % provides \textsubscript
\ifnum 0\ifxetex 1\fi\ifluatex 1\fi=0 % if pdftex
  \usepackage[T1]{fontenc}
  \usepackage[utf8]{inputenc}
\else % if luatex or xelatex
  \ifxetex
    \usepackage{mathspec}
    \usepackage{xltxtra,xunicode}
  \else
    \usepackage{fontspec}
  \fi
  \defaultfontfeatures{Mapping=tex-text,Scale=MatchLowercase}
  \newcommand{\euro}{€}
\fi
% use upquote if available, for straight quotes in verbatim environments
\IfFileExists{upquote.sty}{\usepackage{upquote}}{}
% use microtype if available
\IfFileExists{microtype.sty}{\usepackage{microtype}}{}
\usepackage{longtable,booktabs}
\ifxetex
  \usepackage[setpagesize=false, % page size defined by xetex
              unicode=false, % unicode breaks when used with xetex
              xetex]{hyperref}
\else
  \usepackage[unicode=true]{hyperref}
\fi
\hypersetup{breaklinks=true,
            bookmarks=true,
            pdfauthor={},
            pdftitle={},
            colorlinks=true,
            citecolor=blue,
            urlcolor=blue,
            linkcolor=magenta,
            pdfborder={0 0 0}}
\urlstyle{same}  % don't use monospace font for urls
\setlength{\parindent}{0pt}
\setlength{\parskip}{6pt plus 2pt minus 1pt}
\setlength{\emergencystretch}{3em}  % prevent overfull lines
\setcounter{secnumdepth}{0}


\begin{document}

\section{Database design}\label{database-design}

Database records financial transactions with the following data:

\begin{itemize}
\itemsep1pt\parskip0pt\parsep0pt
\item
  Date
\item
  Payee
\item
  Category
\item
  Description
\item
  Amount
\end{itemize}

It would be useful to track individual items within a transaction, for
example if shopping, as different purchases in a single transaction may
have very different categories. A separate purchase table should
therefore be used.

\subsection{Transactions Table}\label{transactions-table}

The transactions table is the core of the database, and has a form as
following:

\begin{longtable}[c]{@{}lllll@{}}
\toprule\addlinespace
transaction\_id & date & payee\_id & description & amount
\\\addlinespace
\midrule\endhead
1 & 2015-01-02 & 1 & Train ticket & 46.85
\\\addlinespace
2 & 2015-01-03 & 2 & Shopping & 5.42
\\\addlinespace
\bottomrule
\end{longtable}

The columns are:

\begin{longtable}[c]{@{}lll@{}}
\toprule\addlinespace
Column name & Data type & Description
\\\addlinespace
\midrule\endhead
transaction\_id & SERIAL & Unique ID for transaction (also serves as
primary key)
\\\addlinespace
date & DATE & Date of transaction
\\\addlinespace
payee\_id & BIGINT UNSIGNED NOT NULL & Unique ID of PAYEE (or payer) -
relates to PAYEES table
\\\addlinespace
description & VARCHAR(100) & Variable length string describing the
transaction
\\\addlinespace
amount & DECIMAL(10,2) & Value of transaction - negative for expense,
positive for income
\\\addlinespace
\bottomrule
\end{longtable}

The SERIAL data type is an alias for BIGINT UNSIGNED NOT NULL
AUTO\_INCREMENT UNIQUE and is therefore very useful for PRIMARY KEYS.

The DECIMAL(10,2) data type is an ``exact'' floating point number with
10 digits, two decimal places (meaning up to 8 digits before the decimal
place).

\subsection{Transaction Item Table}\label{transaction-item-table}

The transaction item table has 2 purposes: 1. For transactions with
multiple purchases or items, these can be broken down into the
sub-transactions (e.g.~different items while shopping). 2. Transactions
may be categorised using the breakdown table. The breakdown table
relates transactions to categories, through sub-purchases if these
exist. This allows different items in a single purchase to receive
different categories. The table has this form:

\begin{longtable}[c]{@{}lllll@{}}
\toprule\addlinespace
transaction\_item\_id & transaction\_id & description & amount &
category\_id
\\\addlinespace
\midrule\endhead
1 & 1 & Train ticket: Aberdeen to Cambridge & 46.85 & 3
\\\addlinespace
2 & 2 & Apples & 3.00 & 4
\\\addlinespace
3 & 2 & Milk & 1.00 & 5
\\\addlinespace
4 & 2 & Bread & 1.42 & 5
\\\addlinespace
\bottomrule
\end{longtable}

The columns are:

\begin{longtable}[c]{@{}lll@{}}
\toprule\addlinespace
Column name & Data type & Description
\\\addlinespace
\midrule\endhead
transaction\_item\_id & SERIAL & Unique ID for transaction (PRIMARY KEY)
\\\addlinespace
transaction\_id & BIGINT UNSIGNED NOT NULL & Links to transaction in
TRANSACTIONS table
\\\addlinespace
description & VARCHAR(100) & Variable length string describing item
\\\addlinespace
amount & DECIMAL(10,2) & Value of item - negative for expense, positive
for income
\\\addlinespace
category id & BIGINT UNSIGNED NOT NULL & Unique ID of category - relates
to CATEGORIES table
\\\addlinespace
\bottomrule
\end{longtable}

\subsection{Payees table}\label{payees-table}

The payees table contains information about payees, which is referenced
by their unique IDs in the transactions table.

\begin{longtable}[c]{@{}ll@{}}
\toprule\addlinespace
payee\_id & payee\_name
\\\addlinespace
\midrule\endhead
1 & National Rail
\\\addlinespace
2 & Sainsbury's
\\\addlinespace
\bottomrule
\end{longtable}

The columns are:

\begin{longtable}[c]{@{}lll@{}}
\toprule\addlinespace
Column name & Data type & Description
\\\addlinespace
\midrule\endhead
payee\_id & SERIAL & Unique ID for Payee (PRIMARY KEY)
\\\addlinespace
payee\_name & VARCHAR(100) & Name of payee
\\\addlinespace
\bottomrule
\end{longtable}

\subsection{Categories table}\label{categories-table}

The categories table contains categories/tags for the transactions.
These allow spending to be tracked by different budget areas. Categories
are grouped hierarchically through parent categories.

\begin{longtable}[c]{@{}lll@{}}
\toprule\addlinespace
category\_id & parent\_id & category\_name
\\\addlinespace
\midrule\endhead
1 & NULL & travel
\\\addlinespace
2 & NULL & food
\\\addlinespace
3 & 1 & train
\\\addlinespace
4 & 2 & fruit
\\\addlinespace
5 & 2 & staple
\\\addlinespace
\bottomrule
\end{longtable}

The columns are:

\begin{longtable}[c]{@{}lll@{}}
\toprule\addlinespace
Column name & Data type & Description
\\\addlinespace
\midrule\endhead
category\_id & SERIAL & Unique ID for category (PRIMARY KEY)
\\\addlinespace
parent\_id & BIGINT UNSIGNED & ID of parent category
\\\addlinespace
category\_name & VARCHAR(100) & Category name
\\\addlinespace
\bottomrule
\end{longtable}

\end{document}
